\documentclass[12pt,a4paper,oneside]{article}
\usepackage{hyperref}
\usepackage[utf8]{vietnam}
\usepackage{multirow}
\usepackage[english]{babel}
\usepackage{freecontest}
\usepackage{verbatim}
\usepackage{graphicx}
\usepackage{float}

\header{\LARGE Free Contest 118}

\begin{document}

\problemtitle{MEDIAN}

\renewcommand{\baselinestretch}{1.25}
\setlength{\parskip}{1em}

Với một dãy số $B$ gồm $K$ phần tử $b_1, b_2, \ldots, b_K$, ta định nghĩa trung vị của dãy $B$ như sau: Gọi $B'$ là dãy số nhận được từ $B$ sau khi sắp xếp theo thứ tự tăng dần. Khi đó, phần tử thứ $\left \lfloor \frac{K}{2} \right \rfloor + 1$ của dãy $B'$ là trung vị của dãy $B$.

Ví dụ:
\begin{itemize}
\item Phần tử trung vị của dãy $[3, 7, 5, 4]$ là $5$.
\item Phần tử trung vị của dãy $[1, 7, 13, 4, 1]$ là $4$.
\item Phần tử trung vị của dãy $[100]$ là $100$.
\end{itemize}

Cho dãy số $A$ gồm $N$ phần tử $a_1, a_2, \ldots, a_N$. Với mỗi cặp chỉ số $(l, r)$ sao cho $1 \le l \le r \le N$, gọi $m_{l, r}$ là phần tử trung vị của dãy $a_l, a_{l+1}, \ldots, a_r$. Ta sẽ liệt kê giá trị $m_{l, r}$ với tất cả các cặp $(l, r)$ để tạo thành dãy $M$. Hãy cho biết phần tử trung vị của dãy $M$.

\heading{Dữ liệu}

\setlength{\parskip}{0.25em}

\begin{itemize} 
\item Dòng đầu tiên gồm số nguyên dương $N$ ($1 \le N \le 10^5$) là số phần tử của dãy $A$.
\item Dòng thứ hai gồm $N$ số nguyên $a_1, a_2, \ldots, a_N$ ($1 \le a_i \le 10^9$).
\end{itemize}

\heading{Kết quả}

\begin{itemize}
\item In ra phần tử trung vị cần tìm.
\end{itemize}

\heading{Ví dụ}

\renewcommand{\baselinestretch}{1.0}
\begin{example}
    \exmp{
3
20 5 10
  }{%
10
  }%
    \exmp{
5
30 30 30 30 30
  }{%
30
  }%
    \exmp{
7
1 3 7 2 5 4 6
  }{%
5
  }%
\end{example}

\heading{Giải thích}

Ở ví dụ thứ nhất:

\begin{itemize}
\item Dãy $[20]$ có trung vị là $20$.
\item Dãy $[5]$ có trung vị là $5$.
\item Dãy $[10]$ có trung vị là $10$.
\item Dãy $[20, 5]$ có trung vị là $20$.
\item Dãy $[5, 10]$ có trung vị là $10$.
\item Dãy $[20, 5, 10]$ có trung vị là $10$.
\end{itemize}

Do đó, dãy $M$ là $[20, 5, 10, 20, 10, 10]$ và trung vị của dãy $M$ là $10$.

\heading{Chấm điểm}

\begin{itemize}
\item Subtask 1 (20\% số điểm): $N \leq 200$
\item Subtask 2 (30\% số điểm): $N \leq 2000$
\item Subtask 3 (50\% số điểm): Không có giới hạn gì thêm
\end{itemize}

\end{document}